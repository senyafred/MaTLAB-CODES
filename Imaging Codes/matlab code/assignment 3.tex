\documentclass[10pt,a4paper]{article}
\usepackage[latin1]{inputenc}
\usepackage{amsmath}
\usepackage{amsfonts}
\usepackage{amssymb}
\usepackage{graphicx}
\begin{document}
\subsection*{Deriving the Mean of for continuous amplitude Gaussian}
\[\int_{-\infty}^{\infty}\frac{1}{\sqrt{2\pi\sigma^2}}\exp\left\{-\frac{(x-\mu)^2}{2\sigma^2}\right\}dx=1\]
with respect to the two parameters $\mu$  and $\sigma$ (RHS will then be zero).
\\The Gaussian pdf is defined as
\[f_X(x) =\frac{1}{\sigma\sqrt{2\pi}}\exp\left\{-\frac{(x-\mu)^2}{2\sigma^2}\right\}\]
where\,\, $\mu$ and $\sigma$ are two parameters, with $\sigma >0$. By definition of the mean we have
\[E(X) = \int_{-\infty}^{\infty}x\frac{1}{\sigma\sqrt{2\pi}}\exp\left\{-\frac{(x-\mu)^2}{2\sigma^2}\right\}dx\]
which using integral properties can be written as

\begin{align*}
E(X)& = \int_{-\infty}^{\infty}(x+\mu)\frac{1}{\sigma\sqrt{2\pi}}\exp\left\{-\frac{x^2}{2\sigma^2}\right\}dx\\
&=\int_{-\infty}^{\infty}x\frac{1}{\sigma\sqrt{2\pi}}\exp\left\{-\frac{x^2}{2\sigma^2}\right\}dx \;+\; \int_{-\infty}^{\infty}\mu\frac{1}{\sigma\sqrt{2\pi}}\exp\left\{-\frac{x^2}{2\sigma^2}\right\}dx\\
\textrm{For the first integral, call }& \textrm{it}\,\, I_1 \,\,\textrm{we have using additivity}\\
I_1& = \int_{-\infty}^0x\frac{1}{\sigma\sqrt{2\pi}}\exp\left\{-\frac{x^2}{2\sigma^2}\right\}dx + \int_{0}^{\infty}x\frac{1}{\sigma\sqrt{2\pi}}\exp\left\{-\frac{x^2}{2\sigma^2}\right\}dx\\
\textrm{Swapping the integration} & \textrm{ limits in the first we have}\\
I_1 &= -\int_{0}^{-\infty}x\frac{1}{\sigma\sqrt{2\pi}}\exp\left\{-\frac{x^2}{2\sigma^2}\right\}dx + \int_{0}^{\infty}x\frac{1}{\sigma\sqrt{2\pi}}\exp\left\{-\frac{x^2}{2\sigma^2}\right\}dx\\
\textrm{and using again integral} &\,\, \textrm{properties we have}\\
I_1& = \int_{0}^{\infty}(-x)\frac{1}{\sigma\sqrt{2\pi}}\exp\left\{-\frac{(-x)^2}{2\sigma^2}\right\}dx + \int_{0}^{\infty}x\frac{1}{\sigma\sqrt{2\pi}}\exp\left\{-\frac{x^2}{2\sigma^2}\right\}dx\\
\Rightarrow I_1 &= -\int_{0}^{\infty}x\frac{1}{\sigma\sqrt{2\pi}}\exp\left\{-\frac{x^2}{2\sigma^2}\right\}dx + \int_{0}^{\infty}x\frac{1}{\sigma\sqrt{2\pi}}\exp\left\{-\frac{x^2}{2\sigma^2}\right\}dx = 0\\
\textrm{So we have that}\qquad\qquad &\\
E(X)& = \int_{-\infty}^{\infty}\mu\frac{1}{\sigma\sqrt{2\pi}}\exp\left\{-\frac{x^2}{2\sigma^2}\right\}dx\\
\textrm{Multiply by}\,\, \sigma\sqrt{2} \,\,\textrm{to obtain}\\
E(X) &= \int_{-\infty}^{\infty}\mu\frac{1}{\sqrt{\pi}}e^{-x^2} dx = \mu\frac{2}{\sqrt{\pi}}\int_{0}^{\infty}e^{-x^2} dx\\
\textrm{Now}\qquad\qquad\qquad\qquad&\\
\frac{2}{\sqrt{\pi}}\int_{0}^{\infty}e^{-x^2} dx& = \lim_{t\rightarrow \infty}\frac{2}{\sqrt{\pi}}\int_{0}^{t}e^{-x^2} dx = \lim_{t\rightarrow \infty} \text{erf}(t) = 1\\
\textrm{where "erf" is the error}&\textrm{ function So we end up with}\\
E(X) &= \mu\\
\textrm{hence the parameter}&\,\, \mu\,\, \textrm{is the mean of the distribution.}
\end{align*}

\subsection*{Now the Deriving the Variance}
We have
\begin{align*}
\text {Var}(X) &= \int_{-\infty}^{\infty}(x-\mu)^2\frac{1}{\sigma\sqrt{2\pi}}\exp\left\{-\frac{(x-\mu)^2}{2\sigma^2}\right\}dx\\\\
\textrm{Applying the same tricks as before we have}&\textrm{}\\\\
\int_{-\infty}^{\infty}(x-\mu)^2\frac{1}{\sigma\sqrt{2\pi}}\exp\left\{-\frac{(x-\mu)^2}{2\sigma^2}\right\}dx &= \int_{-\infty}^{\infty}x^2\frac{1}{\sigma\sqrt{2\pi}}\exp\left\{-\frac{x^2}{2\sigma^2}\right\}dx\\\\
&=\sigma \sqrt2\int_{-\infty}^{\infty}(\sigma \sqrt2x)^2\frac{1}{\sigma\sqrt{2\pi}}\exp\left\{-\frac{(\sigma \sqrt2x)^2}{2\sigma^2}\right\}dx \\\\
&= \sigma^2\frac{4}{\sqrt{\pi}}\int_{0}^{\infty}x^2e^{-x^2}dx\\\\
\textrm{Define} \,\,t=x^2\Rightarrow x= \sqrt t\,\,\textrm{and}\,\,dt = 2xdx& = 2\sqrt tdx \Rightarrow dx = (2\sqrt t)^{-1}dt\\\\
\textrm{Substituting}\qquad\qquad\qquad\qquad\qquad& \\\\
[V(X) = \sigma^2\frac{4}{\sqrt{\pi}}\int_{0}^{\infty}(\sqrt t)^2(2\sqrt t)^{-1}e^{-t}dt = &
\sigma^2\frac{4}{\sqrt{\pi}}\frac 12 \int_{0}^{\infty}t^{\frac 32 -1}e^{-t}dt\\\\
= &\sigma^2\frac{4}{\sqrt{\pi}}\frac 12 \Gamma\left(\frac 32\right)\\\\
\Rightarrow V(X) =& \sigma^2\frac{4}{\sqrt{\pi}}\frac 12 \frac {\sqrt \pi}{2} = \sigma^2
\end{align*}
where $\Gamma()$ is the Gamma function. So the parameter $\sigma$ is the square-root of the variance, i.e. the standard deviation.

\subsection*{Mean and Variance of a Uniform Distribution}
Using the basic definition of expectation we may write:
\begin{align*}
E(X)& = \int_{-\infty}^{\infty} xf(x)dx = \int_{a}^{b}x \dfrac{1}{b-a}dx=\dfrac{1}{2(b-a)}\left[x^2\right]^b_a\\
&= \dfrac{b^2-a^2}{2(b-a)}\\
&= \dfrac{b+a}{2}
\end{align*}
Using the formula for the variance, we may write:
\begin{align*}
V(X) &= E(X^2)-\left[E(X)\right]^2\\\\
&=\int_{a}^{b}x^2\cdot\dfrac{1}{b-a}dx-\left(\dfrac{b+a}{2}\right)^2=\dfrac{1}{3(b-a)}\left[x^3\right]^b_a-\left(\dfrac{b+a}{2}\right)^2\\\\
&=\dfrac{b^3-a^3}{3(b-a)}-\left(\dfrac{b+a}{2}\right)^2\\\\
&=\dfrac{b^2+ab+a^2}{3}-\dfrac{b^2+2ab+a^2}{4}\\\\
&=\dfrac{(b-a)^2}{12}
\end{align*}


\end{document}